\documentclass{article}
\usepackage{graphicx} % Para insertar imágenes
\usepackage{longtable} % Para tablas largas
\usepackage{array} % Para formatear columnas

\title{COPILADORES}
\author{\\johan Leonardo Márquez Zúñiga \\ Diego Nova Rosas \\ Angélica Castillo Tovar \\ Renzo Murrillo Álvarez}
\date{March 2025}

\begin{document}

\maketitle

\section{Introduction}

El presente trabajo muestra cómo logramos integrar un nuevo lenguaje utilizando como herramienta una librería de Python, la cual nos permite compilar diversos ejemplos, algunos de los cuales presentaremos a lo largo del documento. 

Para llevar a cabo este desarrollo, fue necesario diseñar y estructurar una tabla en la que definimos nuestros propios tokens. Estos tokens fueron diseñados siguiendo una lógica similar a la de los lenguajes C++ y Python, lo que nos permitió establecer una sintaxis coherente y funcional. 

Además, durante el proceso de implementación, analizamos las diferencias y similitudes entre ambos lenguajes con el objetivo de adaptar sus características de manera eficiente en nuestro nuevo lenguaje.

\section{Tabla de Tokens y Expreciones}

\renewcommand{\arraystretch}{1.3} % Aumenta el espaciado entre filas para mejorar la legibilidad

\begin{longtable}{|l|l|}
    \hline
    \textbf{Token} & \textbf{Expresiones} \\
    \hline
    \endfirsthead

    \hline
    \textbf{Token} & \textbf{Expresiones} \\
    \hline
    \endhead

    \hline
    \endfoot

    \hline
    \endlastfoot

    var & r'\#[a-zA-Z\_][a-zA-Z0-9\_]*' \\ \hline
    letter & '[a-zA-Z]+' \\ \hline
    number & '[0-9]+' \\ \hline
    edad & + \\ \hline
    less & - \\ \hline
    mult & * \\ \hline
    div & / \\ \hline
    left\_por & ( \\ \hline
    right\_por & ) \\ \hline
    def COMMENT & r'\textless--.*?--\textgreater' \\ \hline
    while & while \\ \hline
    for & for \\ \hline
    if & if \\ \hline
    switch & switch \\ \hline
    else & else \\ \hline
    else if & r'else' \\ \hline
    break & broken \\ \hline
    punto\_coma & ; \\ \hline
    menor & \textless \\ \hline
    mayor & \textgreater \\ \hline
    igual & = \\ \hline
    menor\_igual & \textless= \\ \hline
    mayor\_igual & \textgreater= \\ \hline
    diferente & \textless|\textgreater \\ \hline
    igual\_que & == \\ \hline
    and & @ \\ \hline
    or & // \\ \hline
    comment & r'\textless--.*?--\textgreater' \\ \hline
    funk\_return & funk\_ret \\ \hline
    funk\_not\_return & funk\_not\_ret \\ \hline
    imprimir & r"imprimir\s*\(\s*'([a-zA-Z0-9_ ])'\s\)\s*;" \\ \hline
    main & main \\ \hline
    inicio & \{ \\ \hline
    fin & \} \\ \hline
    return & ret \\ \hline
    id (para factorial, num, etc) & r'[a-zA-Z\_][a-zA-Z0-9\_]*' \\ \hline
\end{longtable}
\section{Librerías y Análisis Léxico}

\subsection{Importación de Librerías y Definición de Tokens}

\begin{verbatim}
import ply.lex as lex

# Lista de tokens
tokens = ("IGUAL", "NUMBER", "ADD", "LESS", 
"MULT", "DIV", "LEFT_POR", "RIGHT_POR","INICIO", "FIN", 
"VAR", "WHILE","FOR","IF","ELSE_IF", 
"COMMENT","BREAK","PUNTO_COMA", "MENOR",
"MAYOR", "MAYOR_IGUAL", "MENOR_IGUAL",
"IGUAL_QUE","DIFERENTE", "AND", "OR", 
"FUNK_RET", "FUNK_NOT_RET", "MAIN", 
"RETURN", "IMPRIMIR", "ID")
\end{verbatim}

Se importa la librería **PLY (Python Lex-Yacc)** para la construcción del analizador léxico.  definimos  una lista de tokens, que son palabras clave o símbolos del lenguaje personalizado.

---

\subsection{Definición de Expresiones Regulares para Tokens Simples}

\begin{verbatim}
# Expresiones regulares para tokens simples
t_IGUAL = r'\='
t_ADD       = r'\+'
t_LESS      = r'\-'
t_MULT      = r'\.'
t_DIV       = r'/'
t_LEFT_POR  = r'\('
t_RIGHT_POR = r'\)'
t_DIFERENTE = r'\textless{}\textgreater{}'
t_INICIO = r'{'
t_FIN = r'}'
t_MENOR = r'\textless{}'
t_MAYOR = r'\textgreater{}'
t_MENOR_IGUAL = r'\textless{}='
t_MAYOR_IGUAL = r'\textgreater{}='
t_PUNTO_COMA = r';'
t_IGUAL_QUE = r'=='
t_AND = r'@'
t_OR = r'//'
\end{verbatim}

Aquí  definimos las expresiones regulares para reconocer los operadores básicos del lenguaje, como `+`, `-`, `.` (multiplicación en este lenguaje), `\textless{}\textgreater{}` (diferente), y `@` (operador lógico AND).

---

\subsection{Definición de Tokens para Palabras Clave}

\begin{verbatim}
#imprimir
def t_IMPRIMIR(t):
    r"imprimir\s*\(\s*'([a-zA-Z0-9_ ])'\s\)\s*;"
    return t

#retornar
def t_RETURN(t):
    r'ret'
    return t

#funciones
def t_FUNK_RET(t):
    r'funk_ret'
    return t

def t_FUNK_NOT_RET(t):
    r'funk_not_ret'
    return t
\end{verbatim}

Estas funciones definen los tokens correspondientes a palabras clave del lenguaje.  
- **"imprimir"**: Se detecta la palabra clave `imprimir()`.  
- **"ret"**: Se usa para el `return`.  
- **"funk_ret" y "funk_not_ret"**: Se identifican las funciones con o sin retorno.

---

\subsection{Tokens para Estructuras de Control y Variables}

\begin{verbatim}
# Expresión regular while
def t_WHILE(t):
    r'while'
    return t

# Expresión regular para variables que empiezan con \#
def t_VAR(t):
    r'\# [a-zA-Z_][a-zA-Z0-9_]*'
    return t

# COMENTARIOS VDM
def t_COMMENT(t):
    r'\texttt{\textless{}--.*?--\textgreater{}}'
    return t
\end{verbatim}

Se definen expresiones regulares para estructuras de control como `while`, variables que comienzan con `\#`, y comentarios delimitados por `\texttt{\textless{}-- ... --\textgreater{}}`.

---

\subsection{Lectura de Código Fuente y Generación de Tokens}

\begin{verbatim}
# Construcción del analizador léxico
lexer = lex.lex()

# Leer el archivo de entrada
archivo = input("Ingrese el nombre del archivo de código fuente: ")

try:
    with open(archivo, "r", encoding="utf-8") as file:
        data = file.read()
except FileNotFoundError:
    print(f"Error: No se encontró el archivo '{archivo}'")
    exit(1)

# Pasar la entrada al lexer
lexer.input(data)

#almacenando los tokens en una lista
tokens_list = []

# Obtener los tokens
print("\nTokens detectados:")
for tok in lexer:
    print(tok)
    tokens_list.append({"tipo": tok.type, "valor": tok.value, "linea": tok.lineno})

# Mostrar los tokens en formato de lista
print("\nTokens detectados en formato de diccionario:")
print(tokens_list)
\end{verbatim}

Aquí creamos el analizador léxico,leemos un archivo de entrada y se generan los tokens.  
- **`lexer.input(data)`** procesa el código.  
- **Se imprimen y almacenan los tokens** en una lista de diccionarios.  

---

**¿Qué cambió en comparación con un lenguaje convencional?**  
- Se usa **`@` para AND** en lugar de `and`.  
- **`.` como multiplicador** en vez de `*`.  
- **`\textless{}\textgreater{}` para "diferente"** en vez de `!=`.  
- Las variables comienzan con `\#`, como en `\#var1`.  
- Los comentarios usan `\texttt{\textless{}-- --\textgreater{}}` en lugar de `#` o `/* */`.  
- Definiciones de funciones con `funk_ret` y `funk_not_ret`.  
- La impresión de texto usa `imprimir()` en vez de `print()`.  


\section{Ejemplos}

\subsection{Ejemplo 1: Cálculo de Factorial con Recursión}

\begin{verbatim}
<-- ejemplo1 -->

funk_ret factorial (#n){
    if(n==0)
        ret 1;
    else
        ret n . factorial(n-1);
}

main () {
    #num=2
    imprimir('resultado');
    imprimir(factorial(num));
}
\end{verbatim}

Este ejemplo define una función llamada `factorial`, la cual calcula el factorial de un número usando recursión.  

- La función `factorial(#n)` recibe un número como parámetro.  

- Si `n` es igual a 0, devuelve 1. En caso contrario, devuelve `n` multiplicado por `factorial(n-1)`.  

- En la función `main()`, se define la variable `#num` con un valor de `2`. Luego, se imprime la palabra "resultado" y el factorial del número ingresado.  

**Cambio:** Se usa `funk_ret` en vez de `int`, `ret` en lugar de `return`, y `.` para multiplicación en vez de `*`.  

---

\subsection{Ejemplo 2: Condicionales y Ciclos}

\begin{verbatim}
<-- ejemplo2 -->

if (#var1 + 10) {
    while (#var2 - 5){
        #var3 = #var1 . #var2;
    }
}
\end{verbatim}

En este fragmento de código, se combinan estructuras condicionales (`if`) y ciclos (`while`).  

- La condición del `if` verifica si `#var1 + 10` es verdadero (es decir, diferente de 0).  

- Si la condición se cumple, se entra en un bucle `while`, el cual continuará ejecutándose mientras `#var2 - 5` sea verdadero.  

- Dentro del bucle, la variable `#var3` toma el valor del producto entre `#var1` y `#var2`.  

**Cambio:** No es necesario `\textgreater 0` en el `if`, y `.` se usa para multiplicación en vez de `*`.  

---

\subsection{Ejemplo 3: Bucles Anidados con Impresión}

\begin{verbatim}
<-- ejemplo3 -->

int main (){
    #i;
    #j;
    for(i=1; i<=2;i++){
        for(j=1;j<=2;j++){
            imprimir('c');
        }
        imprimir(' ');
    }
    ret 0;
}
\end{verbatim}

Este código implementa bucles `for` anidados para imprimir una secuencia de caracteres.  

- Se declaran dos variables `#i` y `#j`.  

- El primer bucle `for(i=1; i<=2; i++)` recorre los valores de `i` de 1 a 2.  

- Dentro de este bucle, hay otro bucle `for(j=1; j<=2; j++)`, el cual imprime la letra `'c'` en cada iteración.  

- Luego de completar el ciclo interno, se imprime un espacio `' '`.  

**Cambio:** imprimir por un cout , ret por un return 



\end{document}

